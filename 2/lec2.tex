\documentclass[10pt]{beamer}

% \usepackage{define}

\usetheme{CCFD}
\usepackage{color}
\definecolor{gray97}{gray}{.90}
\definecolor{gray75}{gray}{.75}
\usepackage{listings}
\lstset{frame=Ltb, showspaces=false,showstringspaces=false,
     framerule=0pt,
     aboveskip=0cm,
     framextopmargin=0pt,
     framexbottommargin=1pt,
     framexleftmargin=0.1cm,
     framesep=1pt,
     rulesep=.4pt,
     backgroundcolor=\color{gray97},
     rulesepcolor=\color{black},
     language=C++,
           basicstyle=\ttfamily\scriptsize,
           keywordstyle=\color{blue}\ttfamily,
           stringstyle=\color{red}\ttfamily,
           commentstyle=\color{black}\ttfamily,
          breaklines=true,
          %
          numbers=left,
          numberstyle=\tiny,
          numbersep=6pt,
%           numberfirstline = false,
%           breaklines=true
          }
\lstdefinestyle{consol}
   {basicstyle=\scriptsize\bf\ttfamily,
    backgroundcolor=\color{gray75},
}
\resetcounteronoverlays{lstnumber}

\newcommand{\tabitem}{%
  \usebeamertemplate{itemize item}\hspace*{\labelsep}}

\usepackage{tikz}
\usetikzlibrary{calc,shapes}

\eventtitle{Computer Science I}
\title{Lecture 2}
\subtitle{data types}
\date{}

\setbeamertemplate{blocks}[rounded][shadow=true]
\setbeamertemplate{navigation symbols}[]

\begin{document}

\frame{
    \titlepage
}

\frame{
  \frametitle{What we have seen so far}
    \begin{itemize}
      \item Basic elements of C
      \item Comments: /* */ - multi line, // - single line
      \item \#include $< >$ or " " - appending headers
      \item Functions
      \item semicolons ; 
      \item Brackets (), \{\}
    \end{itemize}
}

\begin{frame}[fragile]{Program}
    \begin{lstlisting}
/* Description - not mandatory but polite*/
#include <stdio.h> //Preprocessor commands starting with a \# Note no semicolons ";".

void fun2(); // declaration of fun 2, definition below

void fun1() // definition of fun 1
{
  printf("Hello from 1!\n")
}

int main() //The main function must be there
{
  fun1();
  printf("Hello from main!\n", c); // \n starts new line
  fun2();
  return 0;
}

void fun2() // definition of fun2
{
  printf("Hello from 2!\n")
}
   
    \end{lstlisting}
 
\end{frame}

% \begin{frame}[fragile]
%   \frametitle{;}
%   Semicolon ends the line
%   \begin{columns}
%     \begin{column}{0.48\textwidth}
%     \begin{lstlisting}
% x1=(-b-sqrt(delta))/(2*a);
%     \end{lstlisting}
%     \end{column}
%     \begin{column}{0.48\textwidth}
%     \begin{lstlisting}
% x1=(-b
% -sqrt(
% delta))
% /
% (2*a);
%     \end{lstlisting}
%     \end{column}
%   \end{columns}
% \end{frame}

\frame{
  \frametitle{Data types}
  \framesubtitle{Everything lives in computer's memory}

  Our program processes data, and everything it does or uses is stored in computer's memory. 
  Basic data types allow for declaration of variables and allocation of necessary resources (space in memory).
  
  \hspace{1cm}
  
  For today:
    \begin{itemize}
      \item Simple data types with examples.
      \item Arithmetic operations.
      \item Precedence of operators.
      \item Some fun with characters.
      \item Printing of variables with \it printf()
    \end{itemize}
}

\section{Selected data types}

\frame{
  \frametitle{Selected data types in C}
    \begin{itemize}
      \item Used to declare variables or define functions.
      \item Determine size the variable occupies in memory.
      \item Need format specifiers to print with {\it printf()}.
      \item We limit our interest to integers, real numbers, characters, boolean and ... void types
      \item The size defined by a specific data type might vary with implementation (32b vs 64b), use {\it sizeof()}
      \item There is more ...
    \end{itemize}
}

\subsection{int}

\begin{frame}
  \frametitle{Integers}
  \framesubtitle{int, unsigned int, long int, long long int, unsigned long long int}
  \centering

    \begin{itemize}
      \item 4 6 842 -6 2 1024 - integers\\
      \item 4.8 3.141592 1.- not integers - mind the dot
      \item keyword {\bf int}
      \item 2 or {\bf 4} Bytes (B)
      \item $-2,147,483,648 - 2,147,483,647$
      \item Format specifiers for \it printf():
        \begin{itemize}
          \item \%d \%i - signed integer (\%li \%ld for long).
          \item \%o	- Octal integer.
          \item \%x \%X - Hex integer.
          \item \%u - unsigned integer.
        \end{itemize}
      \item 012 in octal representation
      \item 0x10 in hexadecimal representation
    \end{itemize}

\end{frame}

\begin{frame}[fragile]
  \frametitle{int}
  \centering

    \begin{lstlisting}
#include <stdio.h>

int main()
{
  printf("Storage size for int : %ld B \n", sizeof(int));
  int a=032;
  int b=0x23;
  int c=23;
  int d=-1;
  printf("a=%d, b=%d, c=%d, d=%d\n", a, b, c, d);
  printf("a=%x, b=%x, c=%x, d=%x\n", a, b, c, d);
  printf("a=%o, b=%o, c=%o, d=%o\n", a, b, c, d);
  printf("a=%u, b=%u, c=%u, d=%u\n", a, b, c, d);
}
    \end{lstlisting}
\end{frame}

\begin{frame}[fragile]
  \frametitle{int arithmetic}
  \centering

  \begin{columns}
    \begin{column}{0.52\textwidth}
    \begin{lstlisting}
#include <stdio.h>

int main()
{
  int a=5, b=4;
  printf("%d+%d=%d\n", a, b, a+b);
  printf("%d-%d=%d\n", a, b, a-b);
  printf("%d*%d=%d\n", a, b, a*b);
  printf("%d/%d=%d\n", a, b, a/b);
  printf("%d/%d=%d\n", b, a, b/a);
  printf("%d%%%d=%d\n", a, b, a%b);
}
    \end{lstlisting}
    \end{column}
    \begin{column}{0.48\textwidth}
      \begin{itemize}
        \item a + b addition
        \item a - b subtraction
        \item a * b multiplication
        \item a / b division
        \item a \% b remainder of division
      \end{itemize}
      {\small note 1: To print \% one needs \%\%  !\\
      note 2: The result has the same type {\it int} as arguments!}
    \end{column}
  \end{columns}
\end{frame}

\subsection{Real type - floats, doubles}

\begin{frame}
  \frametitle{Real floating-point}
  \framesubtitle{float, double, long double}
  \centering
  4.8 3.141592 1.0 0.5 -3.14 2. - real numbers \\
  
  \begin{columns}
    \begin{column}{0.5\textwidth}
    \centering
      {\large \bf float}
      \begin{itemize}
      \item keyword {\bf float}
      \item 4B (single precision!)
      \item $1.2\cdot10^{-38}$ to $3.4\cdot10^{38}$
      \item 6 decimal places
      \item Format specifiers:
        \begin{itemize}
          \item \%f
        \end{itemize}
    \end{itemize}
    \end{column}
    \begin{column}{0.5\textwidth}
    \centering
      {\large \bf double}
      \begin{itemize}
      \item keyword {\bf double}
      \item 8B (double precision!)
      \item $2.3\cdot10^{-308}$ to $1.7\cdot10^{308}$
      \item 15 decimal places
      \item Format specifiers:
        \begin{itemize}
          \item \%lf
        \end{itemize}
    \end{itemize}
    \end{column}
  \end{columns}
  
  \begin{center}
    \begin{tabular}{@{}l@{}}
        \tabitem \%e \%E	- Scientific notation.\\
        \tabitem \%g \%G - Similar as \%e or \%E.
    \end{tabular}
  \end{center}

\end{frame}

\begin{frame}[fragile]
  \frametitle{float and double}
  \centering

    \begin{lstlisting}
#include <stdio.h>
#include <math.h> // the math library

int main()
{
  printf("Storage size for float : %ld B \n", sizeof(float));
  printf("Storage size for double : %ld B \n", sizeof(double));
  float a = 4.0 * atan(1.0); // This is PI
  double b = 4.0 * atan(1.0); // This is PI
  printf("a=%f, a=%e, a=%E, a=%g, a=%G\n", a, a, a, a, a);
  printf("b=%f, b=%e, b=%E, b=%g, b=%G\n", b, b, b, b, b);
  a = 6.02214085774e23;
  b = 6.02214085774e23;
  printf("a=%f, a=%e, a=%E, a=%g, a=%G\n", a, a, a, a, a);
  printf("b=%f, b=%e, b=%E, b=%g, b=%G\n", b, b, b, b, b);
}
    \end{lstlisting}
\end{frame}

\begin{frame}[fragile]
  \frametitle{float and double arithmetic}
  \centering

  \begin{columns}
    \begin{column}{0.55\textwidth}
    \begin{lstlisting}
#include <stdio.h>

int main()
{
  double a=5.0, b=4.0;
  printf("%lf+%lf=%lf\n", a, b, a+b);
  printf("%lf-%lf=%lf\n", a, b, a-b);
  printf("%lf*%lf=%lf\n", a, b, a*b);
  printf("%lf/%lf=%lf\n", a, b, a/b);
  printf("%lf/%lf=%lf\n", b, a, b/a);
}
    \end{lstlisting}
    \end{column}
    \begin{column}{0.48\textwidth}
      \begin{itemize}
        \item a + b addition
        \item a - b subtraction
        \item a * b multiplication
        \item a / b division
      \end{itemize}
      {note 1: The result has the same type {\it float} or {\it double} as arguments!}
    \end{column}
  \end{columns}
\end{frame}

\subsection{char}

\begin{frame}
  \frametitle{Characters}
  \framesubtitle{char}
  \centering

    \begin{itemize}
      \item 'a' 'b' 'c' '1' etc.
      \item or hexadecimal ASCI code, e.g.: '\textbackslash x15' '\textbackslash x9c' 
      \item keyword {\bf char}
      \item 1B
      \item "a" is not 'a' !  "a" is 'a' and '\textbackslash 0' - null sign
      \item Format specifiers: \%c
    \end{itemize}

\end{frame}

\begin{frame}[fragile]
  \frametitle{char}
  \centering

    \begin{lstlisting}
#include <stdio.h>

int main()
{
  printf("Storage size for char : %ld B \n", sizeof(char));
  char a = 'a';
  printf("a=%c \n", a);
  printf("a=%d \n", a);
  a = '\x15';
  printf("a=%c \n", a);
  printf("a=%d \n", a);
  a = "R";
  printf("a=%c \n", a);
  printf("a=%d \n", a);
  a = 'R';
  printf("a=%c \n", a);
  printf("a=%d \n", a);
  a = '\0';
  printf("a=%c \n", a);
  printf("11 %c 11 %c 11 %c 11 \n", '\0', '\x00', '\x30');
  printf("11 %d 11 %d 11 %d 11 \n", '\0', '\x00', '\x30');
}
    \end{lstlisting}
\end{frame}

\subsection{bool}

\begin{frame}
  \frametitle{bool}
  \framesubtitle{To be full of bool? To be false or true?}
  \centering
    \begin{itemize}
      \item Represents logical value
      \item introduced in C99
      \item need \#include $<$stdbool.h$>$
      \item {\it true} or {\it false}
      \item keyword {\bf bool} or {\bf \_Bool}
      \item 1B or same as int (platform dependant)
      \item Has no format specifier, use \%d (see example)
    \end{itemize}

\end{frame}

\begin{frame}[fragile]
  \frametitle{bool}
  \centering

    \begin{lstlisting}
#include <stdio.h>
#include <stdbool.h>

int main()
{
  printf("Storage size for char : %ld B \n", sizeof(bool));
  bool a = true;
  printf("a=%d\n", a);
  a = false;
  printf("a=%d\n", a);
  
  //For curious students:
  a = true;
  printf("a=%s \n", a ? "true" : "false");
  a = false;
  printf("a=%s \n", a ? "true" : "false");
}
 \end{lstlisting}
\end{frame}

\subsection{void}

\begin{frame}[fragile]
  \frametitle{void}
  \centering
  Specifies no value available
    \begin{itemize}
      \item Functions with no return are {\bf void}
      \item Functions that accept no arguments accepts {\bf void}
      \item There can be a pointer (what is a pointer?) to {\bf void} - it points to an address, but does not specify the variable type - more later on 
      \item keyword {\bf void}
      \item 1B (?) - it has no size
    \end{itemize}
  \centering

    \begin{lstlisting}
#include <stdio.h>

int main()
{
  printf("Storage size for void : %ld B \n", sizeof(void)); //should not work, works in gcc
  void a; //Not possible
  printf(" %d \n", a);
}
    \end{lstlisting}

\end{frame}

\section{Working with variables}

\begin{frame}[fragile]
  \frametitle{Operations on variables}
  \centering
Substitution:
    \begin{lstlisting}
int a,b; //declare two variables of type int
a=35; //assigment: a store value of 35
b=6; //b store 6
    \end{lstlisting}

An arithmetic expresion:
     \begin{lstlisting}   
a=a+b;// perform addition in temporary space, and assign the result to a
    \end{lstlisting}

\end{frame}

\begin{frame}[fragile]
  \frametitle{Assignment}
  \centering
  When using '=' sign variable on the Left of '=' is assigned the value on the Right. This does not necessarily means equality!
  
  The Value on the Right is calculated first and later the value is copied to the Left. Types on the Left and Right should be the same
  and type mixing should be avoided.
  
\begin{lstlisting}
double x1=6.28;
int a = 2
a = x1; //loss of data since a=6!
x1=a;
\end{lstlisting}
There is an explicit way to change the type: casting
\begin{lstlisting}
double x1=6.28;
int a = 2
a = (int)x1; //loss of data, but no warning
x1=(double)2/3; //x1 is not zero
\end{lstlisting}

\end{frame}

\begin{frame}[fragile]
  \frametitle{Precedence of operators}
  \framesubtitle{The ones we know so far}
  \centering
  \begin{enumerate}
    \item () brackets
    \item + - uary plus/minus: (-1)
    \item * / \% binary operator a*b
    \item - + binary operator a+b
  \end{enumerate}

\begin{lstlisting}
-5 * 3 + 4 * 5. / 2. 
((-5)*3)+(4*5)/2.
\end{lstlisting}

\end{frame}

\begin{frame}[fragile]
  \frametitle{Increment/decrement operators}
  Increment / decrement operators are unary operators that change the value of a variable by 1.\\
  They can have postfix or prefix form

\begin{lstlisting}
a++ //postfix
a--
++a //prefix
--a
\end{lstlisting}

\begin{lstlisting}
int a = 1;
int b = a++; // stores 1+a (which is 2) to a
             // returns the value of a (which is 1)
             // After this line, b == 1 and a == 2
a = 1;
int c = ++a; // stores 1+a (which is 2) to a
             // returns 1+a (which is 2)
             // after this line, c == 2 and a == 2
\end{lstlisting}

\end{frame}

\end{document}
